%% LyX 2.2.1 created this file.  For more info, see http://www.lyx.org/.
%% Do not edit unless you really know what you are doing.
\documentclass[english]{beamer}
\usepackage[T1]{fontenc}
\usepackage[latin9]{inputenc}
\setcounter{secnumdepth}{3}
\setcounter{tocdepth}{3}
\usepackage{textcomp}
\usepackage{graphicx}
%\graphicspath{{./output/}}
\usepackage[authoryear]{natbib}
\setbeamertemplate{caption}[numbered]
\usepackage{appendixnumberbeamer} % for additional appendix numbers
\beamertemplatenavigationsymbolsempty % remove navigation symbols

\usepackage[labelfont=bf, textfont = scriptsize]{caption}
\usepackage{amsmath}
\usepackage[scaled]{helvet}


%% EURO Symbol
\usepackage[official]{eurosym}  % euro symbol

%% Subfigures
\usepackage{subcaption}

%% TABLES
\usepackage{multirow}
\usepackage{multicol}
\usepackage{amssymb}

% custom table column with gray background
%\usepackage{xcolor,colortbl}
%\definecolor{Gray}{gray}{0.7}
%\newcolumntype{G}{>{\columncolor{Gray}}c}


%% === check mark
%\usepackage{pifont}
%\newcommand{\cmark}{\ding{51}}
%\newcommand{\xmark}{\ding{55}}

% Gray underline
\newcommand{\light}[1]{\textcolor{gray}{#1}}

% Define colors
\definecolor{darkgreen}{rgb}{0.09, 0.60, 0.27}
\definecolor{bluegray}{rgb}{0.325, 0.525, 0.725}
%\definecolor{bluegraylight}{rgb}{0.475, 0.675, 0.875}
%\definecolor{bluegraylighter}{rgb}{0.55, 0.75, 0.95}
\definecolor{lightgray}{rgb}{0.91, 0.91, 0.91}
\definecolor{darkgray}{rgb}{0.81, 0.81, 0.81}
\definecolor{darkgrayer}{rgb}{0.2, 0.2, 0.2}

\usetheme{CambridgeUS}
\usecolortheme{beaver}
\setbeamercolor*{palette tertiary}{bg=bluegray, fg = white}
\setbeamercolor*{palette secondary}{bg=lightgray, fg = darkgrayer}
\setbeamercolor*{palette primary}{bg=darkgray, fg = darkgrayer}
\setbeamercolor{frametitle}{fg=bluegray,bg=lightgray}
\setbeamercolor{title}{fg=bluegray,bg=white}

%\usecolortheme{seahorse}
%\usecolortheme{default}

%%%%%%%%%%%%%%%%%%%%%%%%%%%%%% Textclass specific LaTeX commands.
 % this default might be overridden by plain title style

%%%%%%%%%%%%%%%%%%%%%%%%%%%%%% User specified LaTeX commands.
% Packages for tables
\usepackage{booktabs}% Pretty tables
\usepackage{threeparttablex}% For Notes below table
\usepackage{graphicx}
\usepackage{tabularx}

% omit table columns
\usepackage{array}
\newcolumntype{H}{>{\setbox0=\hbox\bgroup}c<{\egroup}@{}}
\newcolumntype{Z}{>{\setbox0=\hbox\bgroup}c<{\egroup}@{\hspace*{-\tabcolsep}}}


% For tick and cross marks
\usepackage{pifont}

% Commands for green checkmark and red cross
\newcommand{\cmark}{\textcolor{darkgreen}{\ding{51}}} % Green checkmark
\newcommand{\xmark}{\textcolor{red}{\ding{55}}}   % Red cross
\newcommand{\cmarkparen}{\textcolor{darkgreen}{(\ding{51})}} % Green checkmark with parentheses



\makeatother

\usepackage{babel}

\author[\textbf{Erfort} / Stoetzer]{
\small
\textbf{Cornelius Erfort}, \textit{Witten/Herdecke University} \\ \vspace{0.2cm}
{Lukas F. Stoetzer}, \textit{Witten/Herdecke University} \\ \vspace{0.2cm}
\vspace{-0.5cm}
}

\title[How Swing Model Assumptions Shape Vote-to-Seat Predictions]{How Swing Model Assumptions Shape Vote-to-Seat Predictions: Evidence from Recent German Elections}

%\vspace*{0cm}
 \date[{DVPW AK Wahlen 2025}]{
	DVPW AK Wahlen 2025 \\ \vspace{1.25em}  
	%\textit{Panel: Multinationals and Governments}\\  
	% \vspace{0.25em} February 19, 2025\\\vspace{1.25em}
	    {\footnotesize Supported by\\ \vspace{.5em} Deutsche Forschungsgemeinschaft (DFG \#529544178)} 
}

% font size of captions for whole document
\setbeamerfont{caption}{size=\footnotesize} 

\begin{document}


\begin{frame}
\maketitle	
\end{frame}




\section{Motivation}

\frame{
 \vspace{.4cm}
\centering\fbox{\includegraphics[width=.9\textwidth]{../figures/accuracy.pdf}} \\
    \scriptsize \\ \vspace{-.05cm}
    % \flushright Source: Bundestag, Parlamentsfernsehen

}

\begin{frame}[t]{Swing Models in Electoral Forecasting}
	


\begin{itemize}\setlength\itemsep{0.8em}
	\item \textbf{How do different swing assumptions affect electoral predictions?}
	\begin{itemize}\setlength\itemsep{0.5em}
		\item Extensive research on electoral forecasting {\textcolor{gray}{\scriptsize \citep{bafumi2006forecasting,fisher2011polls,hanretty2016combining}}} \pause
		\item \textcolor{bluegray}{\bfseries Limited focus}: \pause (1) uniform vs. proportional swing comparison, \pause  (2) mixed-member proportional systems \pause and (3) systematic evaluation across parameter space \pause
	\end{itemize}

		\item[] \textbf{$\rightarrow$ \alert{3 Gaps}:}
				\begin{itemize}\setlength\itemsep{0.3em}
                \item[1.] When do different swing assumptions perform optimally?
                \item[2.] How do party system characteristics affect model choice?
                \item[3.] Empirical vs. theoretical insights \pause
		% \item YYY \textbf{lacks} \textbf{ZZZZ} \pause and \textbf{OOO} \pause $\rightarrow$ XXX \alert{PPP},  but \textcolor{bluegray}{III}.
		\end{itemize}
		\pause
\vspace*{0.25cm}
	\item[$\rightarrow$] \textbf{Systematic evaluation of swing models} using \bfseries German electoral data \alert{\&} comprehensive simulation study.\pause
		% \begin{itemize}\setlength\itemsep{0.5em}
		% 	\item[] \textbf{RQ1}: \textit{How do lobbyists compare to the general population?}\pause
		% 	\item[] \textbf{RQ2}: \textit{What predicts becoming a lobbyist in the first place?}
		% \end{itemize}

\end{itemize}


\end{frame}

\section{Theory \& German Context}

\begin{frame}[t]{Swing Model Assumptions}
\noindent \textcolor{bluegray}{\bfseries Two Dominant Approaches}
    \begin{itemize}\setlength\itemsep{1em}
		% \item<2->\textcolor{bluegray}{\bfseries Expertise vs. Connections}
			\item<1-> \alert{\textbf{Uniform Swing}} $=$ \textit{Identical percentage-point shifts across all districts} {\textcolor{gray}{\scriptsize \citep{fisher2011polls}}}
            \item[1.]<2-> \textbf{Proportional Swing} 
            \begin{itemize}\setlength\itemsep{0.7em}
            \item Vote shares change relative to baseline support levels {\textcolor{gray}{\scriptsize \citep{hanretty2016combining}}}
            \end{itemize}
			\item[2.]<3-> \textbf{Hybrid Approaches}
                \begin{itemize}\setlength\itemsep{0.7em}
                    \item Combine elements of both strategies {\textcolor{gray}{\scriptsize \citep{wilson2022models}}}
                    \item Piecewise models allow different swing behavior above/below party averages
                    \item Interaction models capture continuous relationships between baseline support and swing magnitude
                \end{itemize}
	% \item<6-> \alert{\textbf{Germany -- Institutional differences in comparison to U.S.}}
	% 		\begin{itemize}\setlength\itemsep{0.6em}
	% 	\item<7->[1.] \textbf{Role of federal ministries for legislation} $\rightarrow$ importance of previous \underline{experience in federal government}, rather than parliament 
	% 	\item[8->[2.] \textbf{Public campaign finance system} $\rightarrow$ importance of \underline{personal connections} for access, rather than campaign contributions
	% 	\item<9->[3.] \textbf{Mixed-member proportional representation} $\rightarrow$ importance of \underline{party connections}, rather than connections to individual legislators
	% \end{itemize}
	\end{itemize}
\end{frame}

\section*{Data}

\begin{frame}{German Federal Election Data}

\begin{itemize}\setlength\itemsep{1em}
    \item<1->[1.] \textcolor{black}{\bfseries District-Level Election Results (1998-2025)}
    \begin{itemize}\setlength\itemsep{0.25em}
        \item 8 federal elections, 299 electoral districts
        \item More than 20,000 observations
        \item First votes (district representatives) and second votes (party lists)
    \end{itemize}
    \item<2->[2.] \textcolor{bluegray}{\bfseries \alert{Out-of-Sample Forecasting Framework}}
    \begin{itemize}\setlength\itemsep{0.25em}
        \item Train on 7 elections, predict the 8th
        \item Rotate through all possible combinations
        \item Mirrors real-world forecasting conditions
    \end{itemize}
    \item<3-> \textcolor{bluegray}{\bfseries Candidate and District Covariates}
    \begin{itemize}\setlength\itemsep{0.25em}
        \item Candidate characteristics: incumbency, gender, education, list position
        \item District characteristics: East Germany, number of candidates
        \item Historical performance: past vote shares, previous candidacies
        \item \textit{Comprehensive control for systematic factors}
    \end{itemize}
\end{itemize}
\end{frame}

\begin{frame}{Model Variants: Ten Distinct Specifications}

\begin{table}[htbp]
    \centering\small
    \resizebox{0.9\textwidth}{!}{%
    \begin{tabular}{p{3cm} p{6cm} p{5cm}}
        \toprule
        \textbf{Model Type} & \multicolumn{1}{c}{\textbf{Specification}} & \multicolumn{1}{c}{\textbf{Key Features}} \\
        \midrule
        \textbf{Proportional} & \texttt{proportional\_swing} + covariates & Past second vote share plus proportional change \\
        \addlinespace\addlinespace
        \textbf{Uniform} & \texttt{uniform\_swing} + covariates & Past second vote share plus uniform change \\
        \addlinespace\addlinespace
        \textbf{Mixed} & Both swing terms + covariates & Regression finds optimal balance \\
        \addlinespace\addlinespace
        \textbf{Piecewise} & Swing + interaction terms & Different effects above/below party average \\
        \addlinespace\addlinespace
        \textbf{Pure} & Only swing terms & No candidate/district covariates \\
        \addlinespace\addlinespace
        \textbf{Interaction} & Continuous interactions & \texttt{res\_l1\_Z*swing\_term} \\
        \addlinespace\addlinespace
        \textbf{No Adjustment} & All covariates, no swing & Baseline control model \\
        \bottomrule
    \end{tabular}
    }
    % \caption{Comparison of Registration Rules for Lobby Registers in Germany and the United States}
    \label{tab:model_variants}
\end{table}
    
\end{frame}

\section{Results}

\begin{frame}{Empirical Analyses}
    \begin{enumerate}\setlength\itemsep{1.5em}
        \item<1->[1.] \textbf{Model Performance}: Accuracy differences across swing assumptions
        \begin{itemize}\setlength\itemsep{0.25em}
            \item[a)] District winner prediction accuracy
            \item[b)] Mean error of vote share predictions
            \item[c)] Out-of-bounds predictions
        \end{itemize}
        % \item<2->[2.] \alert{\bfseries Cross-Sectional Analysis}: Predicting becoming a lobbyist.
        \item<2->[2.] \textbf{\textcolor{bluegray}{Simulation Study}}: Systematic parameter space exploration
        \begin{itemize}\setlength\itemsep{0.25em}
            \item 20,251 unique scenario combinations
            \item Party system characteristics
            \item Electoral volatility effects
        \end{itemize}
        \end{enumerate}
\end{frame}

\begin{frame}[t]{Empirical Results: Model Performance}

% \vspace{-0.5cm}
\begin{figure}[t]
    \centering
         \centering
        \includegraphics<1>[width=0.9\textwidth]{../figures/accuracy.pdf}%
        \includegraphics<2>[width=0.9\textwidth]{../figures/mispredicted_results_connected_facets.pdf}%
\end{figure}
\vspace{-0.5cm}


\begin{itemize}
    \item<2->[$\rightarrow$] \textbf{Small but meaningful differences}: Less than 1 percentage point between best (mixed) and worst (proportional piecewise) models. \pause \alert{Interaction terms} significantly reduce prediction errors.
\end{itemize}
    
\end{frame}

\begin{frame}[t]{Empirical Results: Out-of-Bounds Predictions}

% \vspace{-0.5cm}
\begin{figure}[t]
    \centering
         \centering
        \includegraphics<1>[width=0.8\textwidth]{../figures/all_years_incorrect_forecasts.pdf}%
        \includegraphics<2>[width=0.8\textwidth]{../figures/coefplot_correct_comparison_custom.pdf}%
\end{figure}
\vspace{-0.5cm}


\begin{itemize}
    \item<2->[$\rightarrow$] \textbf{Regional clustering} of mispredicted districts suggests local factors not captured by national swing models. \pause \alert{Pure proportional models} perform worst on out-of-bounds predictions.
\end{itemize}
    
\end{frame}

\begin{frame}[t]{Simulation Results: Parameter Space Exploration}

% \vspace{-0.5cm}
\begin{figure}[t]
    \centering
         \centering
        \includegraphics<1>[width=0.825\textwidth]{../figures/01_parameter_grid.pdf}%
        \includegraphics<2>[width=0.825\textwidth]{../figures/04_key_interaction.pdf}%
\end{figure}
\vspace{-0.5cm}


\begin{itemize}
    \item<2->[$\rightarrow$] \textbf{Key interaction}: Increasing parties favors uniform swing, but more small parties favors proportional swing. \pause \alert{Party system fragmentation} affects optimal model choice.
\end{itemize}
    
\end{frame}

\begin{frame}[t]{Simulation Results: First Differences Analysis}

% \vspace{-0.5cm}
\begin{figure}[t]
    \centering
         \centering
        \includegraphics<1>[width=0.8\textwidth]{../figures/05_regression_coefficients.pdf}%
        \includegraphics<2>[width=0.8\textwidth]{../figures/06_first_differences.pdf}%
\end{figure}
\vspace{-0.5cm}


\begin{itemize}
    \item<2->[$\rightarrow$] \textbf{Standardized effects}: Uniform share parameter significantly increases uniform swing advantage. \pause \alert{Higher national variance} benefits proportional swing, \pause \alert{higher district volatility} favors uniform swing.
\end{itemize}
    
\end{frame}

\begin{frame}{Key Findings Summary}
	
	\begin{itemize}[<+->]\setlength\itemsep{1.5em}
		\item \textbf{Empirical Analysis of German Elections}
            \begin{itemize}\setlength\itemsep{0.5em}
                \item Differences between swing models are generally small but meaningful \pause
                \item Interaction terms significantly improve prediction accuracy and reduce out-of-bounds predictions \pause
                \item Regional clustering suggests local factors remain important despite national trends
            \end{itemize}
            \item \textbf{Simulation Study Insights}
            \begin{itemize}\setlength\itemsep{0.5em}
                \item Uniform swing performs better with more parties and higher electoral volatility
                \item Proportional swing performs better with many small parties
                \item Party system characteristics systematically affect optimal model choice
            \end{itemize}
            \item \textbf{Contributions}:
                \begin{itemize}\setlength\itemsep{0.5em}
                \item[1.] Nuanced understanding of when different swing assumptions are most appropriate
                \item[2.] Comprehensive evaluation framework for electoral forecasting across diverse contexts
                \end{itemize}
	\end{itemize}

\end{frame}

\begin{frame}
\vspace*{1cm}
	\centering \Huge Thank You \\ \vspace*{1cm}
 \large \alert{\textbf{Cornelius Erfort}}: \href{cornelius.erfort@uni-wh.de}{cornelius.erfort@uni-wh.de} \\\vspace*{0.75cm}
 \large \textbf{Lukas F. Stoetzer}: \href{lukas.stoetzer@uni-wh.de}{lukas.stoetzer@uni-wh.de} \\\vspace*{0.75cm}
%	\large \textbf{Web}: \href{www.janstuckatz.com}{janstuckatz.com} \\
\end{frame}	
	


\begin{frame}[allowframebreaks]{References} 
	\footnotesize 
	\bibliographystyle{apsr}
	\bibliography{../references}
\end{frame}




\appendix
\section{Appendix}

\begin{frame}[t]{Additional Empirical Results}

% \vspace{-0.5cm}
\begin{figure}[t]
    \centering
         \centering
        \includegraphics<1->[width=0.8\textwidth]{../figures/coefplot_error_comparison_custom.pdf}%
\end{figure}
\vspace{-0.5cm}


% \begin{itemize}
%     \item<2->[$\rightarrow$] 
% \end{itemize}
    
\end{frame}

\begin{frame}[t]{Additional Simulation Results}

% \vspace{-0.5cm}
\begin{figure}[t]
    \centering
         \centering
        \includegraphics<1->[width=0.8\textwidth]{../figures/02_model_performance_heatmap.pdf}%
\end{figure}
\vspace{-0.5cm}


% \begin{itemize}
%     \item<2->[$\rightarrow$] 
% \end{itemize}
    
\end{frame}



\end{document}